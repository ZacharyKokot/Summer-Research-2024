\documentclass[12pt]{article}

\usepackage[letterpaper]{geometry} % For setting the margins

\geometry{
    left=30mm,
    right=30mm,
    top=30mm,
    bottom=40mm,
    headheight = 15pt
}

\usepackage{fancyhdr} % For setting the header and footer
 % Creates a fancy header.
 \pagestyle{fancy}
 \fancyhf{}
 \rhead{\thepage}
 \renewcommand{\headrulewidth}{0pt} % Removes the horizontal line in the headers


\usepackage{tcolorbox} % For colored boxes

\newtcolorbox[auto counter]{theorem}{colback=blue!5!white,colframe=blue!75!black,fonttitle=\bfseries,fontupper=\itshape,title=Theorem~\thetcbcounter}

\newtcolorbox[auto counter]{lemma}{colback=yellow!5!white,colframe=yellow!75!black,fonttitle=\bfseries,fontupper=\itshape,title=Lemma~\thetcbcounter}

\author{Zachary Kokot}
\title{Supplementary Information}

\begin{document}
    \maketitle

    \tableofcontents

    \newpage

    \section{Proof any Density Matrix can be Expressed as a Linear Combination of Pauli Matrices}

    \section{Proof of the Sample Complexity for the Random Local Pauli Measurement Primitive}

    \subsection{Motivation}
    Part of what makes shadow tomography appealing is that it requires a smaller number of measurements to predict an observable within the same error when compared to full state tomography. In this section, we will prove that the random local Pauli measurement primitive requires a number of measurements that scales logarithmically with the number of target observables and exponentially in the locality of the observables. This is a significant improvement over full tomography, which requires a number of measurements that scales exponentially with the number of qubits.

    \subsection{Theorem}
    To estimate the expectation value of $M$ observables by median of means estimation with maximum error $\epsilon$ and success probability $1-\delta$ we turn to the results of 

    \subsection{Proof}

    \section{Proof of the Computational Complexity for the Estimation of Pauli Observables from shadows generated by the Random Local Pauli Measurement Primitive}

    \subsection{Motivation}
    The computational complexity of the estimation of Pauli observables from shadows generated by the random local Pauli measurement primitive is an important consideration when determining the feasibility of the primitive. In this section, we will prove that the computational complexity of the estimation of Pauli observables from shadows generated by the random local Pauli measurement primitive scales linearly with the number of qubits, the number of target observables, the size of the shadow.

    \subsection{Theorem}

    \subsection{Proof}

\end{document}